\documentclass{article}

% Packages
\usepackage{graphicx} % for including images
\usepackage{amsmath} % for math equations
\usepackage{hyperref} % for hyperlinks

% Title and author
\title{My LaTeX Document}
\author{John Doe}

\begin{document}

\maketitle

\begin{abstract}
This is the abstract of my document.
\end{abstract}

\section{Introduction}
This is the introduction of my document.

\section{Equations}
Here is an example equation:
\begin{equation}
    f(x) = x^2
\end{equation}

\section{Figures}
Here is an example figure:
\begin{figure}[h]
    \centering
    \includegraphics[width=0.5\textwidth]{example-image}
    \caption{This is an example figure.}
    \label{fig:example}
\end{figure}

\section{Conclusion}
This is the conclusion of my document.

\begin{thebibliography}{9}
\bibitem{latexcompanion} 
Michel Goossens, Frank Mittelbach, and Alexander Samarin. 
\textit{The \LaTeX\ Companion}. 
Addison-Wesley, Reading, Massachusetts, 1993.
 
\bibitem{einstein} 
Albert Einstein. 
\textit{Zur Elektrodynamik bewegter K{\"o}rper}. (German) 
[\textit{On the electrodynamics of moving bodies}]. 
Annalen der Physik, 322(10):891–921, 1905.
\end{thebibliography}

\end{document}
